% Kale Ewasiuk (kalekje@gmail.com)
% 2021-09-20
%
% Copyright (C) 2021 Kale Ewasiuk
%
% Permission is hereby granted, free of charge, to any person obtaining a copy
% of this software and associated documentation files (the "Software"), to deal
% in the Software without restriction, including without limitation the rights
% to use, copy, modify, merge, publish, distribute, sublicense, and/or sell
% copies of the Software, and to permit persons to whom the Software is
% furnished to do so, subject to the following conditions:
%
% The above copyright notice and this permission notice shall be included in
% all copies or substantial portions of the Software.
%
% THE SOFTWARE IS PROVIDED "AS IS", WITHOUT WARRANTY OF
% ANY KIND, EXPRESS OR IMPLIED, INCLUDING BUT NOT LIMITED
% TO THE WARRANTIES OF MERCHANTABILITY, FITNESS FOR A
% PARTICULAR PURPOSE AND NONINFRINGEMENT.  IN NO EVENT
% SHALL THE AUTHORS OR COPYRIGHT HOLDERS BE LIABLE FOR
% ANY CLAIM, DAMAGES OR OTHER LIABILITY, WHETHER IN AN
% ACTION OF CONTRACT, TORT OR OTHERWISE, ARISING FROM,
% OUT OF OR IN CONNECTION WITH THE SOFTWARE OR THE USE
% OR OTHER DEALINGS IN THE SOFTWARE.

\documentclass{article}
\usepackage{url}
\begin{document}

    \section*{Penlight -- Lua libraries for use in LuaLaTeX}
    v. 2021-09-20, Kale Ewasiuk, \url{kalekje@gmail.com}\\\\

        The official documentation for the Lua library can be found here:\\
  \mbox{\url{https://stevedonovan.github.io/Penlight/api/manual/01-introduction.md.html#}}
    \\

    \subsection*{Required Package Option}
    The first option sent to this package MUST be one of: \\
    \texttt{[penlight]} \ \ \  or \ \ \  \texttt{[pl]}.\\
    All Penlight sub-modules are then available under this global variable by either\\
    \texttt{penlight.XYZ} or \texttt{pl.XYZ}



    \subsection*{Additional Package Options}

    \noindent
    \begin{tabular}{lp{4.5in}}
    \texttt{stringx} & will import additional string functions into the string meta table.\\
                    & this will be ran in pre-amble: \texttt{require('pl.stringx').import()}\\
                        & \hspace*{-4em}\url{https://stevedonovan.github.io/Penlight/api/libraries/pl.stringx.html}\\\\
    \texttt{format} & allows \% operator for Python-style string formating\\
            & this will be ran in pre-amble: \texttt{require('pl.text').format\_operator()}\\
                & \hspace*{-4em}\url{https://stevedonovan.github.io/Penlight/api/libraries/pl.text.html}\\\\
    \texttt{func} & allows placehold expressions eg. \texttt{\_1+1} to be used \\
                & this will be ran in pre-amble: \texttt{penlight.utils.import('pl.func')}\\
                & \hspace*{-4em}\url{https://stevedonovan.github.io/Penlight/api/libraries/pl.func.html}\\\\
    \texttt{extras} & adds some additional functions. Adds functions to some Penlight sub-modules. The following Lua globals will be defined:
                \texttt{close\_bkt\_cnt,
                    add\_bkt\_cnt,
                    reset\_bkt\_cnt,
                    \_NumBkts,
                \_xTrue,
                \_xFalse,
                \_xNoValue,
                \_\_SKIP\_TEX\_\_,
                hasval,
                mod, mod2,
                }\\
    \end{tabular}
    \\\\

    If you want to use Penlight (and extras) with the \texttt{texlua} intrepreter (no document made, only for Lua files, useful for testing),
    you can access it by the following:
     \begin{verbatim}
    package.path = package.path .. ';'..'path/to/texmf/tex/latex/penlight/?.lua'
    penlight = require('penlight')
    __SKIP_TEX__ = true  --only required if you want to use
                         --penlightextras without a LaTeX run
    require('penlightextras')
    \end{verbatim}





    \section*{}
    Disclaimer: I am not the author of the Lua Penlight library.
    Penlight is Copyright \textcopyright  2009-2016 Steve Donovan, David Manura.
    The distribution of Penlight used for this library is:
\url{https://github.com/lunarmodules/penlight}\\
    The author of this library has merged all Lua sub-modules into one file for this package.


\end{document}