% Kale Ewasiuk (kalekje@gmail.com)
% +REVDATE+
% Copyright (C) 2021-2022 Kale Ewasiuk
%
% Permission is hereby granted, free of charge, to any person obtaining a copy
% of this software and associated documentation files (the "Software"), to deal
% in the Software without restriction, including without limitation the rights
% to use, copy, modify, merge, publish, distribute, sublicense, and/or sell
% copies of the Software, and to permit persons to whom the Software is
% furnished to do so, subject to the following conditions:
%
% The above copyright notice and this permission notice shall be included in
% all copies or substantial portions of the Software.
%
% THE SOFTWARE IS PROVIDED "AS IS", WITHOUT WARRANTY OF
% ANY KIND, EXPRESS OR IMPLIED, INCLUDING BUT NOT LIMITED
% TO THE WARRANTIES OF MERCHANTABILITY, FITNESS FOR A
% PARTICULAR PURPOSE AND NONINFRINGEMENT.  IN NO EVENT
% SHALL THE AUTHORS OR COPYRIGHT HOLDERS BE LIABLE FOR
% ANY CLAIM, DAMAGES OR OTHER LIABILITY, WHETHER IN AN
% ACTION OF CONTRACT, TORT OR OTHERWISE, ARISING FROM,
% OUT OF OR IN CONNECTION WITH THE SOFTWARE OR THE USE
% OR OTHER DEALINGS IN THE SOFTWARE.


\documentclass[11pt,parskip=half]{scrartcl}
\setlength{\parindent}{0ex}
\newcommand{\llcmd}[1]{\leavevmode\llap{\texttt{\detokenize{#1}}}}
\newcommand{\cmd}[1]{\texttt{\detokenize{#1}}}
\newcommand{\qcmd}[1]{``\cmd{#1}''}
\usepackage{url}
\usepackage{showexpl}
\lstset{explpreset={justification=\raggedright,pos=r,wide=true}}
\setlength\ResultBoxRule{0mm}
\lstset{
	language=[LaTeX]TeX,
	basicstyle=\ttfamily\small,
	commentstyle=\ttfamily\small\color{gray},
	frame=none,
	numbers=left,
	numberstyle=\ttfamily\small\color{gray},
	prebreak=\raisebox{0ex}[0ex][0ex]{\color{gray}\ensuremath{\hookleftarrow}},
	extendedchars=true,
	breaklines=true,
	tabsize=4,
}
\addtokomafont{title}{\raggedright}
\addtokomafont{author}{\raggedright}
\addtokomafont{date}{\raggedright}
\author{Kale Ewasiuk (\url{kalekje@gmail.com})}
\usepackage[yyyymmdd]{datetime}\renewcommand{\dateseparator}{--}
\date{\today}


\RequirePackage[pl,extras]{penlight}
\title{penlight}
\subtitle{Lua libraries for use in LuaLaTeX}

\begin{document}
\maketitle

        The official documentation for the Lua library can be found here:\\
  \mbox{\url{https://lunarmodules.github.io/Penlight}}
    \\

    \subsection*{Required Package Option}
    The first option sent to this package MUST be one of: \\
    \texttt{[penlight]} \ \ \  or \ \ \  \texttt{[pl]}.\\
    All Penlight sub-modules are then available under this global variable by either\\
    \texttt{penlight.XYZ} or \texttt{pl.XYZ}


  
  \subsection*{texlua usage}
If you want to use Penlight (and extras) with the \texttt{texlua} intrepreter (no document made, only for Lua files, useful for testing),
you can access it by setting \cmd{__SKIP_TEX__ = true} and adding the package to path. For example:
 \begin{verbatim}
package.path = package.path .. ';'..'path/to/texmf/tex/latex/penlight/?.lua'
penlight = require('penlight')
-- below is optional
__SKIP_TEX__ = true  --only required if you want to use
                     --penlightextras without a LaTeX run
-- __PL_NO_GLOBALS__ = true -- optional, skip global definitions
require('penlightextras')
\end{verbatim}

\pagebreak


\subsection*{Additional Package Options}

    \noindent
    \hspace*{-6ex}\begin{tabular}{lp{4.8in}}
    \texttt{stringx} & will import additional string functions into the string meta table.\\
                    & this will be ran in pre-amble: \texttt{require('pl.stringx').import()}\\
    \texttt{format} & allows \% operator for Python-style string formating\\
            & this will be ran in pre-amble: \texttt{require('pl.stringx').format\_operator()}\\
      & \mbox{\url{https://lunarmodules.github.io/Penlight/libraries/pl.stringx.html#format_operator}}
    \\
    \texttt{func} & allows placehold expressions eg. \texttt{\_1+1} to be used \\
                & this will be ran in pre-amble: \texttt{penlight.utils.import('pl.func')}\\
    & \url{https://lunarmodules.github.io/Penlight/libraries/pl.func}\\
    \texttt{\llap{extras}noglobals} & does the above three (\cmd{func,stringx,format}); adds some additional functions to \cmd{penlight} module; and adds the \cmd{pl.tex} sub-module.\\
    \texttt{extras} & does the above \texttt{extrasnoglobals} but makes many of the functions global variables.
    \end{tabular}




\subsection*{Extras}

If \cmd{extras} is used, the following Lua globals will be defined:\\

\subsubsection*{Misc stuff}
\llcmd{__SKIP_TEX__} If using package with \cmd{texlua}, set this global before loading \cmd{penlight}\\
\llcmd{__PL_NO_}\cmd{GLOBALS__} If using package with \cmd{texlua} and you don't want to set some globals (described in next sections), set this global before to \cmd{true} loading \cmd{penlight}\\
\llcmd{hasval(x)} Python-like boolean testing\\
\llcmd{COMP'xyz'()} Python-like comprehensions:\\\url{https://lunarmodules.github.io/Penlight/libraries/pl.comprehension.html}\\
\llcmd{math.mod(n,d)}, \cmd{math.mod2(n)} math modulous\\
\llcmd{string.}\cmd{totable(s)} string a table of characters\\
\llcmd{kpairs(t), }\cmd{npairs(t)} iterate over keys only, or include nil value from table ipairs\\

\llcmd{pl.utils.}\cmd{filterfiles}\cmd{(dir,filt,rec)} Get files from dir and apply glob-like filters. Set rec to \cmd{true} to include sub directories\\

\pagebreak

\subsubsection*{\cmd{pl.tex.} module is added}
\llcmd{add_bkt}\cmd{_cnt(n), }\cmd{close_bkt_cnt(n), reset_bkt_cnt} functions to keep track of adding curly brackets as strings. \cmd{add} will return \cmd{n} (default 1) \{'s and increment a counter. \cmd{close} will return \cmd{n} \}'s (default will close all brackets) and decrement.\\
\llcmd{_NumBkts} internal integer for tracking the number of brackets\\
\llcmd{opencmd(cs)} prints \cmd{\cs}\{ and adds to the bracket counters.\\
\\
\llcmd{_xNoValue,}\cmd{_xTrue,_xFalse}: \cmd{xparse} equivalents for commands\\
\\
\llcmd{prt(x),prtn(x)} print without or with a newline at end. Tries to help with special characters or numbers printing.\\
\llcmd{prtl(l),prtt(t)} print a literal string, or table\\
\llcmd{wrt(x), wrtn(x)} write to log\\
\llcmd{help_wrt}\cmd{(s1, s2)} pretty-print something to console. S2 is a flag to help you find.\\
\llcmd{prt_array2d(tt)} pretty print a 2d array\\
\\
\llcmd{pkgwarn}\cmd{(pkg, msg1, msg2)} throw a package warning\\
\llcmd{pkgerror}\cmd{(pkg, msg1, msg2, stop)} throw a package error. If stop is true, immediately ceases compile.\\
\\
\llcmd{defcmd}\cmd{(cs, val)} like \cmd{\gdef}\\
\llcmd{newcmd}\cmd{(cs, val)} like \cmd{\newcommand}\\
\llcmd{renewcmd}\cmd{(cs, val)} like \cmd{\renewcommand}\\
\llcmd{prvcmd}\cmd{(cs, val)} like \cmd{\providecommand}\\
\llcmd{deccmd}\cmd{(cs, dft, overwrite)} declare a command. If \cmd{dft} (default) is \cmd{nil}, \cmd{cs} is set
to a package warning saying \cmd{'cs' was declared and used in document, but never set}. If \cmd{overwrite}
is true, it will overwrite an existing command (using \cmd{defcmd}), otherwise, it will throw error like \cmd{newcmd}.\\



\subsubsection*{global extras}
If \cmd{extras} is used and NOT \cmd{extrasnoglobals}, then some globals are set.\\
All \cmd{pl.tex} modules are added.\\
\cmd{hasval}, \cmd{COMP}, \cmd{kpairs}, \cmd{npairs} are globals.\\
\cmd{pl.tablex} functions are added to the \cmd{table} table.\\

    \section*{}
    Disclaimer: I am not the author of the Lua Penlight library.
    Penlight is Copyright \textcopyright  2009-2016 Steve Donovan, David Manura.
    The distribution of Penlight used for this library is:
\url{https://github.com/lunarmodules/penlight}\\
    The author of this library has merged all Lua sub-modules into one file for this package.


\end{document}